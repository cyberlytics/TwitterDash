\documentclass[conference]{IEEEtran}
\IEEEoverridecommandlockouts
% The preceding line is only needed to identify funding in the first footnote. If that is unneeded, please comment it out.
\usepackage{cite}
\usepackage{amsmath,amssymb,amsfonts}
\usepackage{algorithmic}
\usepackage{graphicx}
\usepackage{textcomp}
\usepackage{xcolor}
\usepackage{hyperref}
\def\BibTeX{{\rm B\kern-.05em{\sc i\kern-.025em b}\kern-.08em
    T\kern-.1667em\lower.7ex\hbox{E}\kern-.125emX}}
\begin{document}

\title{Konzeptpapier\\Twitter-Dash}

% Authoren	
\author{
        \IEEEauthorblockN{Hahn Bastian}
        \IEEEauthorblockA{
                \textit{b.hahn@oth-aw.de}\\
        }
        \and

        \IEEEauthorblockN{Kleber Martin}
        \IEEEauthorblockA{
                \textit{m.kleber2@oth-aw.de}\\
        }
        \and

        \IEEEauthorblockN{Klier Andreas}
        \IEEEauthorblockA{
                \textit{a.klier@oth-aw.de}\\
        }
        \and

        \IEEEauthorblockN{Kreussel Lukas}
        \IEEEauthorblockA{
                \textit{l.kreussel@oth-aw.de}\\
        }
        \and

        \IEEEauthorblockN{Paris Felix}
        \IEEEauthorblockA{
                \textit{f.paris@oth-aw.de}\\
        }
        \and

        \IEEEauthorblockN{Ziegler Andreas}
        \IEEEauthorblockA{
                \textit{a.ziegler1@oth-aw.de}\\
        }
}

\maketitle

\begin{abstract}
        Ziel des Projekts ist es, ein Dashboard zu erstellen, welches Live-Informationen von der Twitter-API abruft und diese visualisiert.
\end{abstract}


\section{Einführung}

%% TODO





% \section{Verwandte Arbeiten}


\section{Anforderungen}

\subsection{Anforderung 1}
Als Twitter-Nutzer möchte ich die Hashtags sehen, die aktuell auf Twitter trenden,
weil ich einen Überblick über das aktuelle Geschehen auf Twitter erhalten will.
\\
AZK:
\begin{itemize}
        \item Twitter-API Call
        \item Aufbereitung der Daten durch Backend
        \item Noch keine Persistenz
        \item Anzeige auf Tabelle in Webseite
\end{itemize}

\subsection{Anforderung 2}
\label{2}
Als Twitter-Nutzer möchte ich den zeitlichen Verlauf über ein ausgewähltes Hashtag sehen,
weil mich die Relevanz eines bestimmten Ereignissen interessiert.
\\
AZK:
\begin{itemize}
        \item Datenhaltung in Datenbank
        \item Periodische (mind.\ 15 min) Abfrage der Twitter-Daten durch Backend und Speicherung
        \item Angabe von Hashtag und Zeitrahmen
        \item Abfrage der Daten aus Datenbank
        \item Anzeige in Diagramm
\end{itemize}

\subsection{Anforderung 3}
Als Twitter-Nutzer möchte ich auch den zeitlichen Verlauf über mehrere ausgewählte Hashtags sehen,
weil mich die Relevanz von mehreren Ereignissen interessiert.
\\
AZK:
\begin{itemize}
        \item Auswahl mehrere Hashtags
        \item Ablauf wie in \ref*{2}
\end{itemize}

\subsection{Anforderung 4}
Als Twitter-Nutzer möchte ich auch nicht durch Hashtags explizit definierte Topics erkennen können,
weil ich einen Überblick über alle diskutierten Themen haben möchte.
\\
AZK:
\begin{itemize}
        \item Gruppierung aller Tweets der letzten Periode von der Twitter-API
        \item Analyse der Tweets mit Topic-Model
        \item Datenhaltung der gefundenen Topics in Datenbank
        \item Anzeige der Topics auf Webseite
\end{itemize}

\subsection{Anforderung 5}
Als Twitter-Nutzer möchte ich die aktuelle Stimmung zu einem Thema sehen,
weil ich das Sentiment zu einem bestimmten Thema haben möchte.
\\
AZK:
\begin{itemize}
        \item Gruppierung der Tweets der letzten Zeitperiode nach Hashtag
        \item Analyse der Tweets mit Sentiment-Analyse
        \item Datenhaltung des Sentiments in Datenbank
        \item Anzeige des Sentiments auf Webseite
\end{itemize}

\subsection{Anforderung 6}
Als Twitter-Nutzer möchte ich einen Tweet angeben können um alle verwandten Informationen dazu sehen zu können.
Als verwandte Informationen gelten:
\begin{itemize}
        \item Zeitlicher Verlauf der im Tweet enthaltenen Hashtags
        \item Sentiment
        \item Topics
\end{itemize}
AZK:
\begin{itemize}
        \item Eingabefeld für URL des Tweets
        \item Abfrage des Tweets über Twitter-API
        \item Sentiment-Analyse des Tweets
        \item Anzeige der verwandten Informationen
\end{itemize}

\subsection{Anforderung 7}
Als Twitter-Nutzer möchte ich sehen,
welche Hashtags besonders oft mit einem ausgewähltem Hashtag zusammen vorkommen,
weil mich der Zusammenhang mit anderen Themen interessiert.
\\
AZK:
\begin{itemize}
        \item Gruppierung der Tweets der letzten Periode nach Hashtag
        \item Zählen der Hashtags der gruppierten Tweets
        \item Datenhaltung der Hashtag-Beziehungspaare
        \item Eingabefeld für Hashtag
        \item Abfrage von Daten aus Datenbank
        \item Anzeige der verwandten Informationen
\end{itemize}

\subsection{Anforderung 8}
Als Twitter-Nutzer möchte ich Twitter-Benutzer abrufen können,
um deren Metadaten angezeigt zu bekommen.
\\
Als Metadaten gelten:
\begin{itemize}
        \item Follower
        \item Tweets
        \item Engagement
        \item Activity
\end{itemize}
AZK:
\begin{itemize}
        \item Eingabefeld für Twitter-Benutzer
        \item Abfrage der Nutzerdaten von Twitter-API
        \item Aufbereitung der Metadaten des Benutzers
        \item Anzeige der Metadaten des Benutzers
\end{itemize}


\section{Methoden}

\subsection*{Datenaquise}

Abfragen der Daten durch die offizielle API von Twitter.

\subsection*{Über die Daten}

In einer lokalen Graphdatenbank werden die aktuellen Filme und Serien diverser Streaminganbieter (Netflix, Amazon Prime, Hulu, Disney+) abgespeichert. Außerdem werden diese durch Informationen aus der IMDB (Internet Movie Database) ergänzt. Mithilfe dieses Datenbestandes soll es möglich sein, komplexe Abfragen zu formulieren. Durch den SPARQL-Endpoint der DBpedia ist der Zugriff auf weitere Details möglich.

\subsection*{Backend}

Als Backend wird ASP.NET Core verwendet. Dieses von Microsoft entwickelte Framework hat sich in der Welt der Microservices in den letzten Jahren etabliert. Über noch nicht definierte Graph Query Languages soll auf die verwendete Graphdatenbank bzw. DBpedia zugegriffen werden können, um Informationen zu den Filmen und Schauspielern auszulesen.

\subsection*{Frontend}

Für die Interaktion mit dem Benutzer wird React verwendet. Über eine REST-Schnittstelle des Backends kann auf benötigte Informationen zugegriffen werden.

%\section*{Referenzen}

\begin{thebibliography}{0}
        \bibitem{cinemate}Cinemate [Online] \url{https://cinemate.me/} (visited on Nov. 15, 2021)
        \bibitem{pickamovieforme}Pickamovieforme [Online] \url{https://pickamovieforme.com/} (visited on Nov. 15, 2021)
        \bibitem{bestsimilar}Bestsimilar [Online] \url{https://bestsimilar.com/} (visited on Nov. 15, 2021)
        \bibitem{tastedive}Tastedive [Online] \url{https://tastedive.com/movies} (visited on Nov. 15, 2021)
        \bibitem{MovieGEN}MovieGEN [Online] \url{http://citeseerx.ist.psu.edu/viewdoc/download?doi=10.1.1.703.4954\&rep=rep1\&type=pdf} (visited on Nov. 15, 2021)
\end{thebibliography}
\vspace{12pt}

\end{document}
